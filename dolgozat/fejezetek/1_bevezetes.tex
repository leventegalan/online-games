\Chapter{Bevezetés}

Fejlődő, változó világban élünk, amelyben fejlődik a játékipar is.
A napjainkban divatos játékokban a szép grafika, és a nagy méretű pályák jelentik a fő szempontot a játékok megítélése szempontjából. A dolgozat ezzel szemben a hagyományos táblás játékokra koncentrál, amelyek így klasszikus értékeket képviselnek a számítógépes játékok között. Ezek segítik a játékost megtanulni koncentrálni, több lépésre előre gondolkodni szórakoztató formában.

A dolgozatban a jól ismert szabályrendszerek kiegészítési módjára szerepel egy sajátos megközelítés. Ezzel lehetőség adódik a játékosoknak személyre szabni az aktuális játékot. A játékosok igényeitől függően így egyszerűbb, vagy bonyolultabb játék lehet a végeredmény. A bevezetett új szabályokkal a játék jellege lényegesen megváltozhat, így az addig alkalmazott stratégiák már nem minden esetben lesznek alkalmazhatók.

A számítógépes játékok egy kézenfekvő változatát jelentik azok az online játékok, amelyek tulajdonképpen webalkalmazások. A bemutatásra kerülő társasjátékok Java Spring keretrendszer és AngularJS segítségével készültek el. A dolgozat részletesen kifejti a játékok kiválasztásánál szereplő szempontokat, a specifikáció, tervezés és fejlesztés lépéseit.
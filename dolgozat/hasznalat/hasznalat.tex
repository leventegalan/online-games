%Az összefoglaló fejezet
\chapter*{Adathordozó használati útmutató}
\addcontentsline{toc}{chapter}{Adathordozó használati útmutató}

%Ebben a fejezetben kell megadnunk, hogy a szakdolgozathoz mellékelt adathordozót (pl. CD) hogyan lehet elérni, milyen strukturát követ. Minimum 1 maximum 4 oldal a terjedelem. Lehet benne több alszakasz is. A fejezet címe nem módosítható, hasonlóan a következõ részhez (Irodalomjegyzék).

\noindent A szakdolgozatomhoz mellékelt adathordozó eszközön található adatok struktúrája:

\begin{itemize}
	\item online-games.zip - Az elkészült alkalmazás tömörített formátumban
	\item dolgozat.zip - A dolgozat forráskódja tömörített formátumban
	\item dolgozat.pdf - A dolgozat .pdf formátumban
	\item manual.txt - telepítési útmutató egyszerű szöveges formátumban
\end{itemize}

\Section{Telepítési útmutató}
Csomagoljuk ki az online-games.zip fájlt a kívánt helyre! \newline

Töltsünk le az Apache TomCat oldaláról a TomCat alkalmazás szerver 9.0.4 verzióját!

Csomagoljuk ki a letöltött fájt, és másoljuk a kívánt helyre!
Végezzük el az alábbi módosításokat a TomCat conf könyvtárában:
Adjuk a tomcat-users.xml fájl végét jelző tag elég az alábbi szöveget:
<role rolename="manager-gui"/>
<user username="admin" password="password" roles="manager-gui"/>
A server.xml fájlban módosítsuk a Connector tagben található port paramétert 9000-re.
Navigáljunk a TomCat alkalmazás szerver bin mappájába, majd indítsuk el a startup.bat fájlt, (linux esetén startup.sh). \newline

Töltsük le a MySQL holnapjáról a Workbench alkalmazás 8.0 CE verzióját, és telepítsük!

Hozzunk létre egy új kapcsolatot (connection) az alábbi adatokkal:

\begin{itemize}
	\item Connection Name: online\_games
	\item Hostname: 127.0.0.1
	\item Port: 3306
	\item Username: og\_manager
	\item Password: onlinegames
	\item Connection Method: Standard (TCP/IP)
\end{itemize}

Csatlakozzunk a létrehozott Connectionhöz.

Futtassuk le a játék mappájában az sql mappában található create-db.sql és fill-db.sql fájlokat.

Győződjünk meg róla, hogy fut az adatbázis szerver (Workbench, Server menü, Server status).

Ha nem fut a szerve, indítsuk el. \newline

A játék mappájában (online-games) találunk egy deployer.cmd fájlt. Nyissuk meg egy szövegszerkesztőben, majd állítsuk be a könyvtárak elérési útvonalait a számítógépünknek megfelelően, majd mentsük el.

\begin{itemize}
	\item server\_dict: Az a mappa, ahol a játk mappájában taláható pom.xml fájl található.
	\item tomcat\_dict: A TomCat alkalmazás szerver webapps mappája, ide kerül a fordítás során generált .war file.
	\item war\_file: A keletkező .war file elérési útvonala, ami a játék (online-games) mappa target könyvátárában található.
	\item tomcat\_file: A tomcat\_dict változóban megadott mappában található .war file elérési útvonala, (ha létezik).
\end{itemize}

Nyissunk egy konzolt, majd navigáljunk a játék mappájába, majd futtassuk le a deployer.cmd fájlt. \newline

Ekkor a program lefordul, és a keletkezett .war fájlt bemásolja a TomCat alkalmazás szerver webapp mappájába. Ezután a TomCat automatikusan feldolgozza a fájt, amelyet nyomon követhetünk a futó startup program konzoljában. Amint a TomCat végzett vele, a játék használatra kész. \newline

Ezek után a játék az alábbi címen válik elérhetővé:

\textit{http://localhost:9000/online-games/WebContent/index.html\#!/login}  \newline

Minden használatkor győződjünk meg, hogy fut az adatbázis szerver és a TomCat szerver!
\Chapter{Bevezetés}

% TODO: Itt egy olyan leírásnak kellene majd szerepelnie, ami felhívja az olvasó figyelmét, hogy
% - milyen fontos, hogy legyen ilyen IDE,
% - jelenleg még nincs ilyen,
% - a feladat elvégzése nem triviális.

% TODO: Címszavasan bemutatásra kell, hogy kerüljenek a felhasznált technológiák.

% 1-2 oldal elég

Fejlődő változó világban élünk, fejlődik a játékipar is, elterjedtek a nagyobb méretű és gépigényű, de annál szebb, nagy pályás, kalandos szerepjátékok. A legtöbben a kalandot, az ingert, a jó grafikát szeretik. Egyre több féle elektronikus eszköz is kerül forgalomba, amelyek szinte mindegyike alkalmas valamilyen játék futtatására.

Miközben hajszoljuk ezeket a nagy, látványos játékokat, szép lassan feledésbe merülnek a régi idők táblajátékai, amikkel anno ugyanúgy órákat voltunk képesek eltölteni. Ugyan kevesebb volt az inger, viszont megtanítottak több lépéssel előre gondolkodni, koncentrálni, mindezt szórakoztató formában. És nem kellettek hozzá drága holmik, némelyikhez elég volt egy toll, és egy darab papír.

Ezeket az elfeledett játékokat szeretném visszahozni új köntösben, némi csavarral, hogy új izgalmakat lehessen felfedezni ezekben is. A játék újdonsága, hogy már nem fixek a szabályok, hanem mi magunk is személyre szabhatjuk őket. Új kihívásokat fedezhetünk fel benne, még több borsot törhetünk ellenfeleink orra alá, és új babérokat is arathatunk. Újra felfedezhetjük gyerekkorunk kedves játékait.
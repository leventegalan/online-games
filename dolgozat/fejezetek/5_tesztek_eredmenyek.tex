\Chapter{Tesztek, eredmények}

% (4-5 oldal)
% TODO: Elkészült dolgok értékelése

% Érdekesebb esetekre példák képernyőképekkel
% Teljesítmény, terhelhetőség vizsgálata
% Egységtesztekre vonatkozóan legalább javaslatok

\Section{Teszt alapú fejlesztés}

Egy alkalmazás elkészítésekor fontos szempont a minőségbiztosítás. Erre manapság rengeteg fejlesztési modellt kitaláltak, melyeknek szerves része a tesztelés. Legtöbb esetben a fejlesztés sorrendje több-kevesebb eltéréssel: tervezés, fejlesztés, tesztelés. Ennek egyik nagy hátránya, hogy ebben a szakaszban már sokkal nehezebb javítani egy hibát, mint az elején, hiszen ekkorra már sokkal nagyobb lehet a program, nehezebb átlátni, és a függőségek miatt több helyen kell változtatni a kódon, nem is beszélve a plusz költségekről. 

Rendkívül fontos tehát a tesztelés minél korábbi elkezdése, sőt, a teszt alapú tervezés. Ez úgy történik, hogy megállapítjuk a feature-öket, vagyis az elvárt funkciókat. Ezek általában már önmagukban is túl nagyok, hogy egyben írják meg, így kisebb részegységekre kell bontani. Ezekhez az egységekhez pontosan megfogalmazott, elfogadhatósági kritériumokat kell állítani. Ezen kritériumok alapján pedig meghatározatók, sőt, el is készíthetők a tesztek is már a tervezés legelején.

Amikor tesztekről beszélünk, főleg a tervezés elején, elsősorban automatizált tesztekről beszélünk. Ennek több előnye is van. Lássunk is néhányat a tervezés korai szakaszában megírt automatizált tesztek előnyei közül a teljesség igénye nélkül:
\begin{itemize}
	\item A tesztek kivitelezéséhez az elkészítés után nincs szükség emberi erőforrásra.
	\item Kevesebb hiba lehetőség a tesztelés során.
	\item Korán felismerhető, így gyorsabban, olcsóbban javítható hibák.
	\item A tesztek minden kódváltoztatás után, fordításkor automatikusan lefutnak.
	\item Információt szolgáltathatnak a feature-ök, komponensek komplexitásáról, a fejlesztéshez szükséges időről és egyéb erőforrásokról. Azokat a funkciókat, amelyekre nehéz tesztet írni, jó eséllyel lefejleszteni is bonyolult lesz.
\end{itemize}

Ezt a tesztelési módszertant acceptance-test-driven developmentnek (ATDD), azaz magyarítva kb. átvételi vizsgálatokon alapuló fejlesztésnek hívják \cite{atdd}.

\Section{Tesztelés az alkalmazásban}

Habár az alkalmazásomba nem kerültek be automatizált tesztek, fontos szempont volt, hogy egy-egy nagyobb egység is kisebb, önmagukban is tesztelhető egységekből épüljön fel. Ezt az elvet főként az amőba frontendes direktíva kódjában tekinthetjük meg, ahol a működést kisebb egységekre, függvényekre bontottam, és az alkalmazás többi részében is ez a követendő példa. A függvények csak egy-egy kisebb feladatot látnak el, és pontosan meghatározható, hogy milyen bemenetre milyen kimenetet kell adniuk.

\Section{Eredmények}

Az alapos tervezésnek, és implementálásnak köszönhetően elkészült egy bemutatható változat, amely tartalmaz egy teljes, működő adatbázist, a szervert, ami a frontenddel és az adatbázissal is kommunikál, és a frontend, ami szépen megjeleníti a tartalmakat a felhasználónak, tartalmaz autentikációt, és elvégzi a kommunikációt a szerverrel.  A játékok közül az amőba került implementálásra, amelyhez két extra szabályváltozat készült, amelyekből kiválaszthatunk akár egyet, akár kettőt.

A fejlesztés során a legnagyobb nehézségek forrása a konfiguráció, és a kommunikáció helyes működésének megteremtése volt, vagyis az alkalmazás alapjainak elkészítése. Konfigurálni kellett az alap szerver alkalmazást, annotálni a komponenseket, beállítani a dispatcher szervletet, amely a kéréseket a megfelelő kontrollerhez irányítja, hozzá kellett adni az adatbázist, meg kellett alapozni a frontendet is. Kezdetben úgy indult, hogy a backen és a frontend külön projektben szerepeljen, külön futó alkalmazásokként érjék el egymást. Ebben az állapotban szükség volt még egy egyszerű külön szervert biztosítani a frontendnek is. Később a backend és a frontend összevonásra került, így a frontendes kis szerverre már nem volt szükség.

Egy másik, ami valamelyest kötődik a konfigurációhoz is, az adatkonverzió. Az alkalmazás rengeteg adatkonverziót tartalmaz. Egy kérés előtt az adatokat ellenőrizni kell, és megfelelő formába kell rendeznünk. A frontend az adatokat JSON objektumként küldi el. Amikor a szerver ezt megkapja, egy kezdetleges Java objektummá kell alakítania, hogy kinyerhesse belőle az információkat. A kinyert adatok alapján a szerver összegyűjti a műveletek elvégzéséhez szükséges objektumokat, és elvégzi a kívánt műveleteket. Ezek során több adatbázis műveletre is szükség lehet, melyek során az objektumot a Java és az ER modellnek megfelelő formátumoknak megfelelően kell konvertálni. A Java objektumok az ER modellel ellentétben nem egyszerű típusú azonosítót használnak, hanem a hivatkozó táblát reprezentáló objektum típusát, amelyeket nagyon körültekintően kellett annotálni, konfigurálni, különben nem működtek. Továbbá az idegen kulcsoknak nem csak a hivatkozó, hanem a hivatkozott osztályban is szerepelni kellett. Mivel ezek egymásra mutatnak, így könnyen előfordulhat, hogy egy végtelen hivatkozás sorozatot indítunk, amely hibás működést eredményez. Ezek miatt gondoskodni kellett róla, hogy a válasz (JSON) objektumba csak a megfelelő adattagok kerüljenek be, csak az egyik objektum hivatkozzon a másikra.

Az adatkonverziók, adatformátumok meghatározását nagyban segítette és gyorsította a tervezés fázisban előre meghatározott REST API.

























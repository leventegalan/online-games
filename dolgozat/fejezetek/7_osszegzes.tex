\Chapter{Összefoglalás}

% TODO: Röviden értékelni kellene, hogy mi és hogy készült el a dolgozat megírása során.

% 1-2 oldal, elég csak a végén megírni.

A dolgozat és a program megírásához sok féle tudásra volt szükség, rengeteg tapasztalatot és gyakorlati készséget adott. Meg kellett ismerni a technológiákat, melyeknek egy részét kezdő szinten ismertem, némely részével pedig korábban egyáltalán nem találkoztam. Szükség volt a jó tervezés készségének elsajátítására is, mert bár elsőnek nem tűnik bonyolultnak az alkalmazás, időnként meglepően komplex problémákat kellett megoldani. A legbonyolultabb részek a konfiguráció, az adatkonverziók, és a folyamatok szinkronizációja volt, hiszen minden kérést, folyamatot számos rétegen kellett végig vinni. 

A dolgozat megírása során sikerült működőképest programot tervezni, és ez alapján egy működőképes demót összerakni. A dolgozat során tervezett funkciók közül ugyan nem készült el mind. Ezek nem problémaként jelennek meg, hanem tervezetten egy későbbi fejlesztési fázis részei lesznek majd.
Az alkalmazás tartalmaz regisztrációt, bejelentkezést, a bejelentkezésnek megfelelően változó menüsort. A korábban tervezett játékok közül az amőba készült el kettő pluszban választható játékszabály módosítással. Elkészültek a rangsorok is, így nyomon követhető, hogy kik a legjobb játékosok, és a felhasználók képet kaphatnak saját teljesítményükről is.







